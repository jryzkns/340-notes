\documentclass{article}
\usepackage[utf8]{inputenc}
\usepackage[margin=0.75in]{geometry} % lots more margin
\pagenumbering{gobble} % ignore page numbers

\usepackage{titling}
\setlength{\droptitle}{-0.75in}

\setlength{\parindent}{0cm}

\usepackage{enumitem}
\usepackage{graphicx}
\usepackage{amsmath}
\usepackage{amsfonts}
\usepackage{hyperref} % for nice looking urls
\usepackage{booktabs} % for making tables
\usepackage{amssymb}
\usepackage{listings}
\usepackage{graphicx}
\usepackage{caption}
\usepackage{subfigure}
\usepackage{multicol}

\usepackage{titlesec}

\titleformat*{\section}{\large\bfseries}

\begin{document}

\begin{multicols*}{2}

\section*{MATH340 Review Notes}

By Jack `jryzkns' Zhou

\section{Axiomatic Approach to $\mathbb{N}$}

\textbf{Definition}: $\mathbb{N}$ is a set with 3 axioms (sometimes referred to as the "Peano Axioms"):

\begin{enumerate}
    \item $1 \in \mathbb{N}$
    \item For every $a \in \mathbb{N}$, there is an element called the \textit{successor} of $a$, written as $succ(a) = a+1 \in \mathbb{N}$
    \item Every element $a \in \mathbb{N}$ arises in this manner: \[\mathbb{N} = \{succ^{k}(1) \; | \; k \geq 0\}\]
\end{enumerate}

\section{Mathematical Induction}

\textbf{Principle of Mathematical Induction}: 

Suppose $X \subseteq \mathbb{N}$ and:
\begin{enumerate}
    \item $1 \in X$
    \item $a \in X \Rightarrow a+1 \in X$
\end{enumerate}
Then $X = \mathbb{N}$.

This is taken as an axiom and cannot be proven from the 3 axioms presented in section 1.

\textbf{Strong Induction}:

Suppose $X \subseteq \mathbb{N}$ satusfies the properties

\begin{enumerate}
    \item $1 \in X$
    \item $\forall i \in [1,n] \;\; i \in X \Rightarrow n+1 \in X$
    
\end{enumerate}

This variant of induction is logically equivalent to the simple form of induction, but in a proof it may be desirable to refer to more than 1 case that is taken to be true, in which case a strong induction is preferred.

\textbf{The Well-Ordering Principle (WP)}:

Every non-empty subset $Y \subseteq \mathbb{N}$ has a minimal element.

We can use WP to prove the Principle of Induction:

Suppose $X \subseteq \mathbb{N}$ has the properties $1 \in X$ and $k \in X \Rightarrow k+1 \in X$, WTS $X \in \mathbb{N}$. Suppose $Y = \{ n \in \mathbb{N} \;|\; n \notin X\}$, then $X = \mathbb{N} \Leftrightarrow Y = \varnothing$

We proceed to show that $Y = \varnothing$ by contradiction, assuming $Y \neq \varnothing$. By WP, $Y$ has a minimum element $n^{*} \in Y$. As $1 \notin Y$ (because $1 \in X$), $n^* > 1$ so $n^* - 1 \in \mathbb{N}$ and $n^* - 1 \notin Y$ because $n^*$ is the minimal element of $Y$. Therfore $n^* - 1 \in X$, but then $succ(n^* - 1) = n^* - 1 + 1 = n^* \in X$ by the inductive hypothesis. As $n^* \in Y$, we have come to a contradiction, and therefore $Y = \varnothing$ and $X = \mathbb{N}$.

\textit{Note}: WP is false for other sets of numbers. For example, there is no minimal element in $\mathbb{R}^{+}$ as $\forall x \in \mathbb{R}^{+}\;\; \frac{1}{2}x < x$.

\section{Operations on $\mathbb{N, Z}$}

\textbf{Multiplication on $\mathbb{N}$}:

Inductively defined with $1 \cdot a := a$ as the base case. If $n \cdot a$ is defined, then $(n+1) \cdot a := n \cdot a + a$.

The Peano Axioms imply the following properties:
\begin{itemize}
    \item Commutativity: $ab = ba$
    \item Associativity: $a(bc) = (ab)c$
    \item Distribution over Addition: $a(b+c) = ab + ac$
\end{itemize}

\textbf{Defining $\mathbb{Z}$ from $\mathbb{N}$}

Suppose we want to solve an equation like $x+5 = 2$ in $\mathbb{N}$, there are no solutions, because $x = 2 -5 \notin \mathbb{N}$. Therefore, we need to invent the notion of negative numbers. 

To do this, we can say that $\mathbb{Z}$ is the set $\mathbb{N} \times \mathbb{N} = \{(a, b) \;|\; a, b \in \mathbb{N}\}$ with an equivalence relation $(a, b) = (a+c, b+c)$ for any $a, b, c \in \mathbb{N}$. The ordered tuple $(a, b)$ represents $a - b$. We can see that $(a + c) - (b + c) = a - b$. More concretely, consider $(5, 0) = (6, 1) = (500, 495)$ and $5 - 0 = 6 - 1 = 500-495$. A negative number $-a$ could then be represented as $(0, a)$.

\textbf{Induction in $\mathbb{Z}$}

WP does not apply to $\mathbb{Z}$, so in practice we either treat +ive and -ive numbers separately, or we go by the absolute value of the numbers.

\section{The Division Theorem in $\mathbb{Z}$}

\textbf{Theorem}:

Let $a \in \mathbb{Z}$ and $b \in \mathbb{N}$. Then there exists unique $q \in \mathbb{Z}, r \in (0, b)$ such that $a = qb + r$.

\textbf{Proof}: We proceed in two steps, showing existence then uniqueness.

\textit{Existence}: We have $a \in \mathbb{Z}, b \in \mathbb{N}$, we define \[X = \{ n \in \mathbb{N} \cup \{0\} \;|\; n = a - qb\}\] For some integer $q$. $X$ is nonempty as $a - qb \geq 0$ by choice of $q$. If $a > 0$, we pick $q = 0$. If $a \leq 0$, we pick $q = a$. By WP, $X$ has a minimal element that we will call $r$; $r = a - qb$ for some $q \in \mathbb{Z}$. Since $r \in \mathbb{N} \cup \{0\}$, $r \geq 0$. $r$ also satisfies $r < b$. If $r \geq b$, then $r - b \in X$ as $r - b = (a - qb) - b = a - (q+1)b$. This contradicts minimality of $r$. Rearranging $r = a - qb$ we get $a = qb + r$.

\textit{Uniqueness}: Suppose we have $(q_1, r_1)$ and $(q_2, r_2)$ both satisfying the theorem, WTS $q_1 = q_2$ and $r_1 = r_2$.

We have $a = q_1b + r_1 = q_2b + r_2$ with $r_1, r_2 \in (0, b)$. If we collect the terms with $b$ on one side, we have $(q_1 - q_2)b = r_2 - r_1$. So, $r_2 - r_1$ is a multiple of $b$. Given the constraint $r_1, r_2 \in (0, b)$, we can see that $r_2 - r_1 \in [-(b-1),(b-1)]$. Therefore it is only possible that $r_2 - r_1 = 0$ is a multiple a multiple of $b$. Therefore $r_2 = r_1$ and $(q_1 - q_2)b = 0 \Rightarrow q_1 = q_2$.

\subsection{What if $b < 0$?}

$a = qb + r \Leftrightarrow a = (-q)(-b) + r$. The theorem still works, but $0 \geq r \geq |b|$ needs to be guaranteed.

\section {Divisibility in $\mathbb{Z}$}

\textbf{Definition}: Let $d, a \in \mathbb{Z}$, we say that $d$ divides $a$, written as $d | a$, if $a = qd$ for some $q \in \mathbb{Z}$.

Equivalently: $d$ is a \textit{divisor} of $a$, $a$ is a \textit{multiple} of $d$, or $a$ is \textit{divisible} by $d$.

Some Facts:
\begin{itemize}
    \item $\forall d \in \mathbb{Z}\;\;d | 0$ but $0 \nmid a$ unless $a = 0$.
    \item If $d$ divides $a \neq 0$ then $|d| \leq |a|$. In particular, the set of divisors of a non-zero integer is finite.
    \item $d|a \Leftrightarrow |d| \;|\; |a|$ 
\end{itemize}

\section {GCD in $\mathbb{Z}$}

\textbf{Definition}: Let $a, b \in \mathbb{Z}$, not both 0. The \textit{greatest common divisor} of $a$ and $b$, $\gcd(a, b)$ is the greatest $d \in \mathbb{Z}$ such that $d | a$ and $d | b$.

\textbf{Lemmas}:
\begin{itemize}
    \item $( d | a \land d | b )\rightarrow d|(a-b)$
    \item $( d | (a-b) \land d | b )\rightarrow d|a$    
\end{itemize}

Note that these lemmas mean that if $d$ is a common divisor of $(a, b)$ then it is equivalent to $d$ is a common divisor of $(b, a-b)$; $\gcd(a, b) = gcd(b, a- b)$.

\section{Bezout's Identity in $\mathbb{Z}$}

\textbf{Theorem}:

Let $g = \gcd(a, b)$. Then $g = ax + by$ for some $x, y \in \mathbb{Z}$.

\textbf{Proof}: Suppose we have two sets:
\[\begin{aligned}
D &= \{ d \in \mathbb{Z} \;|\; d|a \land d|b \} \\
I &= \{ ax + by \;|\; x,y \in \mathbb{Z}\}
\end{aligned}\]

$D$ is the set of all common divisors between $a, b$ and $I$ is the set of all integer combinations of $a, b$.

From this we make claim (1): If $d \in D$ and $n \in I$, then $d | n$. In particular, if $n \neq 0$, $|d| \leq |n|$. 

Since $d \in D$, we have $a = q_1d$ and $b = q_2d$ for some $q_1, q_2 \in \mathbb{Z}$. Similarly, since $n \in I$, we have $n = ax + by$ for some $x, y \in \mathbb{Z}$. We can see that $n = ax + by = q_1dx + q_2dy = d(q_1x + q_2y) \Rightarrow d | n$.

Suppose now we look at $I \cap \mathbb{N}$, the integer multiples of $a,b$ that are natural numbers, we let $n^* = \min(I \cap \mathbb{N}) = ax^*+by^*$.

We proceed to make claim (2) that $n^* | a$ and $n^* | a$ (i.e. $n^* \in D$).

Suppose $n^* \nmid a$, we divide $a$ by $n^*$ to get $a = qn^* + r$, $r \in (0, n^*)$. By definition of $n^*$, we see that \[\begin{aligned}
    r &= a - qn^* \\
    &= a - q(ax^* + by^*)\\
    &=a - qax^* + qby^* \\
    &=a(1 - qx^*) + b(qy^*) \\
\end{aligned}\]

This means that $n^* \in I$ and that contradicts the minimality of $n^*$ as $r \in (0, n^*)$.

Finally, we make our last claim (3): $n^* = \max(D) = \gcd(a, b)$. By claim (2), $n^*$ is a common divisor of $a, b$. If $d \in D$ is any other common divisor, then $d \leq n^*$ by claim (1). We can see that $d \leq |d| \leq |n^*| = n^*$.

Therefore, we have two interpretations of $\gcd(a, b)$:
\begin{itemize}
    \item $\gcd(a, b) = \max(D)$
    
    maximal element in set of common divisors
    \item $\gcd(a, b) = \min(I \cap \mathbb{N})$
    
    smallest positive integer combination of $a, b$.
    
\end{itemize}

\section{Euclidean Algorithm}

\textbf{Theorem}:

If $a = qb + r$, then $\gcd(a, b) = \gcd(b, r)$.

\textbf{Proof}:

It is given that $\gcd(a, b) = \gcd(a - b, b)$. As $r = a - qb$, we can consider applying the $a - b$ operation $q$ times: $\gcd(a, b) = \gcd(b, a - qb) = \gcd(b, r)$.

\textbf{Algorithm}

Given: $(a, b)$ with $a > b > 0$ and repeatedly apply division theorem on $(a, b)$. After each division, we replace $a$ with $b$ and $b$ with the remainder of the division:\[\begin{aligned}
    (a, b) \;\;\;& a  = q_1b + r_1 \\
    (b, r_1) \;\;\;& b  = q_2r_1 + r_2 \\
    (r_1, r_2) \;\;\;& r_1  = q_3r_2 + r_3 \\
    \vdots \;\;\;& \vdots \\
    (r_{k-2},r_{k-1}) \;\;\;& r_{k-2} = q_kr_{k-1}+ r_k \\
    (r_{k-1}, r_k) \;\;\;& r_{k-1} = q_{k+1}r_{k}+ 0 \\
    (r_k, 0) \;\;\;& 
\end{aligned}\]

The algorithm stops when we reach a point where the second value in the tuple is $0$, in which case $\gcd(a, b) = r_k$.

This algorithm is guaranteed to terminate as each of the $r_i$ up to terminating $r_k$ are strictly decreasing natural numbers. By WP there is a minimal element to which this procedure will terminate on.

% TODO: talk about how to recover coefficients x, y in ax + by

\section{Factoring of Integers}

\textbf{Definition}:

An integer $p \neq \pm 1$ is said to be \textit{irreducible} if its only divisors are itself and 1 or -1.

\textbf{Definition}:

An integer $p \neq 0, \pm 1$ is said to be \textit{prime} if for some $a, b \in \mathbb{Z}$, $p|ab$ implies $p|a$ or $p|b$.

\textbf{Theorem}:

If $n \in \mathbb{Z}\setminus\{0,1,-1\}$, $n$ is a product of irreducible integers.

\textbf{Proof}:

We will proceed by strong induction. The base case $n=2$ is irreducible. In the general case, suppose $n > 2$ and the theorem is true for all $i \in [2, n]$. If $n$ is irreducible, then we are done. Otherwise, $n = ab$ for some $a, b \in [2, n]$, in which case $2 \geq a, b < n$ and $a, b$ are products of primes. A product of products of primes is still a product of primes.

\section{Euclid's Lemma}

\textbf{Lemma}:

An integer $p$ is prime if and only if $p$ is irreducible.

\textbf{Proof}:

$(\Rightarrow)$: Let $p$ be prime. To show that $p$ is irreducible, suppose $p = ab$ for some $a, b \in \mathbb{Z}$, WTS $a$ or $b$ is $\pm 1$. As $p = ab$, we know that $p|a$ or $p|b$. WLOG, suppose $p|a$. This means that $a = pd$ for some $d \in \mathbb{Z}$. So, $p = ab = (pd)b$. Suppose we divide $p = (pd)b$ by $p$, we get $1 = db$. Thereore, $d, b = \pm 1$; $p$ is irreducible.

$(\Leftarrow)$: Let $p$ be irreducible. To show that $p$ is prime, suppose $p|ab$ for some $a, b \in \mathbb{Z}$ and show $p | a$ or $p | b$. As $p$ is irreducible, $\gcd(p, a)$ is either $1$ or $p$. If $\gcd(p, a) = p$, then $p|a$ and we are done. Otherwise, if $\gcd(p, a) = 1$, then by Bezout's Indentity, we have $1 = px + ay$ for some $x, y \in \mathbb{Z}$. Therefore, \[b = b(px+ay) = pbx + aby\]

Note that $p|ab$ is equivalent to saying $ab = pn$ for some $n \in \mathbb{Z}$, we can substitute this in:\[b = pbx + pny=p(bx+ny)\] From this we can see that $p|b=p(bx+ny)$; $p$ is prime.


\section{Fundamental Theorem of Arithmetic}

\textbf{Theorem}:

Let $n \in \mathbb{Z}\setminus\{0,1,-1\}$, $n$ is a product of prime numbers. Moreover, given two prime factorizations $n = p_1 \cdots p_k = q_1 \cdots q_l$, $k=l$ and it's possible to re-enumerate $q_1,\ldots,q_l$ so that $\forall i\;\; p_i = \pm q_i$.

\textbf{Proof}:

We proceed by showing existence then uniqueness.

\textit{Existence}: This is already proven with a previous theorem showing that if $n \in \mathbb{Z}\setminus\{0,1,-1\}$, $n$ is a product of irreducible integers.By Euclid's Lemma, we can also say that such $n$ is a product of prime integers.

\textit{Uniqueness}: Suppose $n = p_1 \cdots p_k = q_1 \cdots q_l$. We proceed by induction on $k$. For the base case $k = 1$, we have $n = p_1 = q_1$. In the general case, suppose $n = p_1 \cdots p_{k+1} = q_1 \cdots q_l$. We look at $p_1$, since $p_1 |n = q_1 \cdots q_l$, then $p_1|q_i$ for some $i$. Therefore $p_1 = \pm q_i$. Suppose we let let $q_i$ swap indices with $q_1$, then we have $n = p_1 \cdots p_{k+1} = q_1q_2 \cdots q_l = p_1 q_2 \cdots q_l$. We can divide $p_1 \cdots p_{k+1} = p_1 q_2 \cdots q_l$ by $p_1$, which leaves us with $p_2 \cdots p_{k+1} = q_2 \cdots q_l$ and we can repeat this procedure to set $p_j = \pm q_i$ for all remaining $j$ factors in $p_2 \cdots p_{k+1}$.

\section{Modular Arithmetic}

Let $a \mathbb{Z}$ and $m \in \mathbb{Z}\setminus\{0\}$, where $m$ is commonly referred to as the \textit{modulus}, we explore some definitions:

\textbf{Definition}:

The \textit{residue} of $a$ modulo $m$ is the remainder of $a$ when divided by $m$.

\textbf{Definition}:

The \textit{congruence class} of $a$ modulo $m$ is defined as the set \[[a]_m := \{a' \in \mathbb{Z} \;|\; a \equiv a' (\mathrm{mod} \;m)\}\]

We say that $a$ is a \textit{representative} of $[a]_m$. 

Congruence classes under the same modulus $m$ are either equal or disjoint. If $x, x' \in [a]_m$, then $x' - x | m$.

Alternatively, we can generate $[a]_m$ in the following way:\[[a]_m = \{a + km \;|\; k \in \mathbb{Z}\}\]

\textbf{Definition}:

The integers modulo $m$, $\mathbb{Z}/m\mathbb{Z}$, is the set of congruence classes modulo $m$.

\section{Algebra and Operations on $\mathbb{Z}/m\mathbb{Z}$}

\textbf{Definition}:

In $\mathbb{Z}/m\mathbb{Z}$, addition is defined as \[[a]_m+[b]_m = [a+b]_m\]
and multiplication is defined as \[[a]_m \cdot [b]_m = [ab]_m\]

\textbf{Definition}:

An element $x$ is said to be \textit{invertible} if there exists an element $y$ such that $[xy]_m = [1]_m$. We say $x$ and $y$ are multiplicative inverses

\textbf{Definition}:

An element $x$ is said to be a \textit{zero divisor} if $x \neq 0$ and there exists an element $y \neq 0$ such that $[xy]_m = [0]_m$

\section{Theorems about $\mathbb{Z}/m\mathbb{Z}$}

\textbf{Theorem}:

An element of $\mathbb{Z}/m\mathbb{Z}$ cannot be both an invertible element and a zero divisor.

\textbf{Proof}:

Suppose $[a]_m$ is invertible, then there exists $[a'] \in \mathbb{Z}/m\mathbb{Z}$ such that $[a]_m[a']_m = [1]_m$. Suppose that also $[a]_m[b]_m = [0]_m$ for some $ b \in \mathbb{Z}/m\mathbb{Z}$. Suppose we multiply $[a]_m[b]_m = [0]_m$ by $[a']_m$, we have \[\begin{aligned}\relax
[a]_m[b]_m &= [0]_m\\
([a']_m[a]_m)[b]_m &= [a']_m[0]_m\\
[b]_m &= [0]_m
\end{aligned}\]

Therefore, the only possible $b$ that satisfies $[a]_m[b]_m = [0]_m$ is $0$, therefore $[a]_m$ cannot be both invertible and a zero divisor.

\textbf{Theorem}:

In $\mathbb{Z}/m\mathbb{Z}$, multiplicative inverses are unique whenever they exist.

\textbf{Proof}:

Let $[a]_m \in \mathbb{Z}/m\mathbb{Z}$ be an invertible element and suppose $[a]_2$ has two inverses $b_1, b_2$: \[[a]_m[b_1]_m = [1]_m\;\;\;\;[a]_m[b_2]_m = [1]_m\] We can attempt to evaluate $[b_1]_m[a]_m[b_2]_m$:\[([b_1]_m[a]_m)[b_2]_m = [b_2]_m\;\;\;\;[b_1]_m([a]_m[b_2]_m) = [b_1]_m\]

We can see that depending on the order of operations taken, we get either $b_1$ or $b_2$. but as $[b_1]_m[a]_m[b_2]_m$ is always the same value, we can conclude that $b_1 = b_2$.

\textbf{Theorem}:

$[a]_m$ is an invertible class $\Leftrightarrow$ $\gcd(a,m) = 1$

\textbf{Proof}:

$(\Rightarrow)$: If $[a]_m$ is invertible, then there exists $[b]_m \in \mathbb{Z}/m\mathbb{Z}$ such that $[a]_m[b]_m = [1]_m$. We can see that $ab = 1 + km$ for some $k \in \mathbb{Z}$. We can rearrange this to $ab + (-k)m = 1$. As the $\gcd(a,m)$ is the smallest natural number to $ax+my$, we have $ab + (-k)m = 1$ so $\gcd(a,m) = 1$.

$(\Leftarrow)$: The same logic applies but in the reverse direction.

\textbf{Theorem}:

$[a]_m$ is a zero divisor $\Leftrightarrow$ $\gcd(a,m) > 1$

\textbf{Proof}:

$(\Rightarrow)$: If $[a]_m$ is a zero divisor, $[a]_m$ is not invertible (by previous theorem: an element cannot be both invertible and zero divisor). So $\gcd(a,m) \neq 1$. As $\gcd(a,m) \in \mathbb{N}$ and $\gcd(a,m) \neq 1$, $\gcd(a,m) > 1$.

$(\Leftarrow)$: Suppose $\gcd(a, m) = g > 1$, let $b = m/g$. If we take the product $ab = a \cdot (m/g) = m \cdot (a/\gcd(a,m))$, we can see that $ab|m$ or $[a]_m[b]_m = [0]_m$; $[a]_m$ is a zero divisor.

\textbf{Theorem}:

If $[a] \in \mathbb{Z}/m\mathbb{Z}$ is invertible and $[b] \in \mathbb{Z}/m\mathbb{Z}$ is arbitrary, then the equation \[[a]_m x = [b]_m\] has exactly one solution. Namely, $x = [a]^-1_m[b]_m$.

\textbf{Proof}:

We can show the uniqueness of the solution by solving:\[\begin{aligned} \relax
    [a]_m x &= [b]_m\\
    ([a]^{-1}_m[a]_m) x &= [a]^{-1}_m[b]_m\\
    [1]_m x &= [a]^{-1}_m[b]_m\\
\end{aligned}\]

Therefore, $x$ must be $[a]^{-1}_m[b]_m$.

\section{Invertible elements in $\mathbb{Z}/m\mathbb{Z}$}

\textbf{Definition}:

The set of invertible elements of $\mathbb{Z}/m\mathbb{Z}$ is denoted as $(\mathbb{Z}/m\mathbb{Z})^\times$ or $(\mathbb{Z}/m\mathbb{Z})^*$: \[(\mathbb{Z}/m\mathbb{Z})^\times = \{[a]_m \in \mathbb{Z}/m\mathbb{Z}\;|\; \gcd(a,m) = 1\}\]

\textbf{Definition}:

The cardinality of $(\mathbb{Z}/m\mathbb{Z})^\times$ is denoted $\phi(m)$, where $\phi$ is the Euler Totient function. $\phi$ has the following properties:
\begin{itemize}
    \item[(1)] If $\gcd(a, b) = 1$, then $\phi(a,b) = \phi(a)\phi(b)$
    \item[(2)] If $p$ is prime and $k \geq 1$, then $\phi(p^k) = (p-1)p^{k-1}$
\end{itemize}

We proceed to prove these properties of $\phi$:

\textbf{Proof}(taken from HW3 Q2):

% TODO: inject proof for In $\mathbb{Z}/mn\mathbb{Z}$ where $\gcd(m,n) = 1$, the element $[a]_{mn}$ is invertible if and only if $[a]_m$ and $[a]_n$ are invertible as well.
(1): In $\mathbb{Z}/mn\mathbb{Z}$ where $\gcd(m,n) = 1$, the element $[a]_{mn}$ is invertible if and only if $[a]_m$ and $[a]_n$ are invertible as well. Therefore, each invertible element in $(\mathbb{Z}/mn\mathbb{Z})^\times$ correspond to a pair of elements in $(\mathbb{Z}/m\mathbb{Z})^\times \times (\mathbb{Z}/n\mathbb{Z})^\times$; there is a bijection betwen $(\mathbb{Z}/m\mathbb{Z})^\times$ and $(\mathbb{Z}/m\mathbb{Z})^\times \times (\mathbb{Z}/n\mathbb{Z})^\times$. 

Therefore $\phi(mn) = \#((\mathbb{Z}/mn\mathbb{Z})^\times) = \#((\mathbb{Z}/m\mathbb{Z})^\times \times (\mathbb{Z}/n\mathbb{Z})^\times) = \#((\mathbb{Z}/m\mathbb{Z})^\times) \cdot \#((\mathbb{Z}/n\mathbb{Z})^\times) = \phi(m)\phi(n)$.

(2): We wish to count all elements in $\mathbb{Z}/p^k\mathbb{Z}$ that are invertible. Since we are working with a prime number $p$, it would be easier to count all zero-divisors instead. Namely, only the values $p,2p,3p,...$ divide into $p^k$ since $p$ is prime. Out of a total of $p^k$ classes, every $p$\textsuperscript{th} class is a zero divisor. Therefore there are $p^{k}/p = p^{k-1}$ zero divisors. Taking this amount ($p^{k-1}$) out of the total ($p^k$), we have $\phi(p^k) = p^k - p^{k-1} = (p-1)p^{k-1}$

\textbf{Theorem}:

If $[a]_m \in \mathbb{Z}/m\mathbb{Z}$ is invertible, then $[a]_m^{\phi(m)} = [1]_m$. Note that this implies that $[a]_m^{\phi(m)-1} = [a]_m^{-1}$ because $[a]_m^{\phi(m)-1} [a]_m = [a]_m^{\phi(m)} = [1]_m$. In the case where $m$ is a prime number $p$, we have \textit{Fermat's Little Theorem}:

If $p$ is prime and $p \nmid a$, then $[a]_p^{p-1} = [1]_p$ in $\mathbb{Z}/p\mathbb{Z}$.

\textbf{Proof}:

% TODO: inject proof for if [a], [b] are invertible, so is [ab]

Given $(\mathbb{Z}/m\mathbb{Z})^\times$, we multiply each element by $[a]_m$. If we get $[a]_m[a_i]_m = [a]_m[a_j]_m$, then we multiply by $[a]_m^{-1}$, ensuring every element in $[a]_m \cdot(\mathbb{Z}/m\mathbb{Z})^\times$ is distinct. If two elements are invertible, their product is invertible as well. Therefore, $[a]_m \cdot(\mathbb{Z}/m\mathbb{Z})^\times$ is just $(\mathbb{Z}/m\mathbb{Z})^\times$ with rearranged elements. Now we take the product over $[a]_m \cdot(\mathbb{Z}/m\mathbb{Z})^\times$:\[\begin{aligned}
\prod_{i \in [a]_m \cdot(\mathbb{Z}/m\mathbb{Z})^\times} i &= [a]_m ^{\#((\mathbb{Z}/m\mathbb{Z})^\times)} \prod_{i \in (\mathbb{Z}/m\mathbb{Z})^\times} i \\
&= [a]_m ^{\phi(m)} \prod_{i \in (\mathbb{Z}/m\mathbb{Z})^\times} i\\
&= [a]_m ^{\phi(m)} [1]_m \\
&= [a]_m ^{\phi(m)} \\
&= [1]_m
\end{aligned}\]

Notice that as $(\mathbb{Z}/m\mathbb{Z})^\times$ contains both invertible elements are their inverses, the product over every element in $(\mathbb{Z}/m\mathbb{Z})^\times$ would simply equal to $[1]_m$. From this we can see that $[a]_m ^{\phi(m)} = [1]_m$.

% TODO: continue on sept 28 notes

\end{multicols*}
\end{document}