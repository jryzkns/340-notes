\documentclass{article}
\usepackage[utf8]{inputenc}
\usepackage[margin=0.75in]{geometry} % lots more margin
\pagenumbering{gobble} % ignore page numbers

\usepackage{titling}
\setlength{\droptitle}{-0.75in}

\title{MATH 340 review notes}
\author{jryzkns}
\date{}

\setlength{\parindent}{0cm}

\usepackage{enumitem}
\usepackage{graphicx}
\usepackage{amsmath}
\usepackage{amsfonts}
\usepackage{hyperref} % for nice looking urls
\usepackage{booktabs} % for making tables
\usepackage{amssymb}
\usepackage{listings}
\usepackage{graphicx}
\usepackage{caption}
\usepackage{subfigure}
\usepackage{multicol}

\begin{document}

\maketitle

\begin{multicols*}{2}

\section{Axiomatic Approach to $\mathbb{N}$}

\textbf{Definition}: $\mathbb{N}$ is a set with 3 axioms (sometimes referred to as the "Peano Axioms"):

\begin{enumerate}
    \item $1 \in \mathbb{N}$
    \item For every $a \in \mathbb{N}$, there is an element called the \textit{successor} of $a$, written as $succ(a) = a+1 \in \mathbb{N}$
    \item Every element $a \in \mathbb{N}$ arises in this manner: \[\mathbb{N} = \{succ^{k}(1) \; | \; k \geq 0\}\]
\end{enumerate}

\section{Mathematical Induction}

\textbf{Principle of Mathematical Induction}: 

Suppose $X \subseteq \mathbb{N}$ and:
\begin{enumerate}
    \item $1 \in X$
    \item $a \in X \Rightarrow a+1 \in X$
\end{enumerate}
Then $X = \mathbb{N}$.

This is taken as an axiom and cannot be proven from the 3 axioms presented in section 1.

\textbf{Strong Induction}:

Suppose $X \subseteq \mathbb{N}$ satusfies the properties

\begin{enumerate}
    \item $1 \in X$
    \item $\forall i \in [1,n] \;\; i \in X \Rightarrow n+1 \in X$
    
\end{enumerate}

This variant of induction is logically equivalent to the simple form of induction, but in a proof it may be desirable to refer to more than 1 case that is taken to be true, in which case a strong induction is preferred.

\textbf{The Well-Ordering Principle (WP)}:

Every non-empty subset $Y \subseteq \mathbb{N}$ has a minimal element.

We can use WP to prove the Principle of Induction:

Suppose $X \subseteq \mathbb{N}$ has the properties $1 \in X$ and $k \in X \Rightarrow k+1 \in X$, WTS $X \in \mathbb{N}$. Suppose $Y = \{ n \in \mathbb{N} \;|\; n \notin X\}$, then $X = \mathbb{N} \Leftrightarrow Y = \varnothing$

We proceed to show that $Y = \varnothing$ by contradiction, assuming $Y \neq \varnothing$. By WP, $Y$ has a minimum element $n^{*} \in Y$. As $1 \notin Y$ (because $1 \in X$), $n^* > 1$ so $n^* - 1 \in \mathbb{N}$ and $n^* - 1 \notin Y$ because $n^*$ is the minimal element of $Y$. Therfore $n^* - 1 \in X$, but then $succ(n^* - 1) = n^* - 1 + 1 = n^* \in X$ by the inductive hypothesis. As $n^* \in Y$, we have come to a contradiction, and therefore $Y = \varnothing$ and $X = \mathbb{N}$.

\textit{Note}: WP is false for other sets of numbers. For example, there is no minimal element in $\mathbb{R}^{+}$ as $\forall x \in \mathbb{R}^{+}\;\; \frac{1}{2}x < x$.

\section{Operations on $\mathbb{N, Z}$}

\textbf{Multiplication on $\mathbb{N}$}:

Inductively defined with $1 \cdot a := a$ as the base case. If $n \cdot a$ is defined, then $(n+1) \cdot a := n \cdot a + a$.

The Peano Axioms imply the following properties:
\begin{itemize}
    \item Commutativity: $ab = ba$
    \item Associativity: $a(bc) = (ab)c$
    \item Distribution over Addition: $a(b+c) = ab + ac$
\end{itemize}

\textbf{Defining $\mathbb{Z}$ from $\mathbb{N}$}

Suppose we want to solve an equation like $x+5 = 2$ in $\mathbb{N}$, there are no solutions, because $x = 2 -5 \notin \mathbb{N}$. Therefore, we need to invent the notion of negative numbers. 

To do this, we can say that $\mathbb{Z}$ is the set $\mathbb{N} \times \mathbb{N} = \{(a, b) \;|\; a, b \in \mathbb{N}\}$ with an equivalence relation $(a, b) = (a+c, b+c)$ for any $a, b, c \in \mathbb{N}$. The ordered tuple $(a, b)$ represents $a - b$. We can see that $(a + c) - (b + c) = a - b$. More concretely, consider $(5, 0) = (6, 1) = (500, 495)$ and $5 - 0 = 6 - 1 = 500-495$. A negative number $-a$ could then be represented as $(0, a)$.

\textbf{Induction in $\mathbb{Z}$}

WP does not apply to $\mathbb{Z}$, so in practice we either treat +ive and -ive numbers separately, or we go by the absolute value of the numbers.

\section{The Division Theorem in $\mathbb{Z}$}

\textbf{Theorem}:

Let $a \in \mathbb{Z}$ and $b \in \mathbb{N}$. Then there exists unique $q \in \mathbb{Z}, r \in (0, b)$ such that $a = qb + r$.

\textbf{Proof}: We proceed in two steps, showing existence then uniqueness.

\textit{Existence}: We have $a \in \mathbb{Z}, b \in \mathbb{N}$, we define \[X = \{ n \in \mathbb{N} \cup \{0\} \;|\; n = a - qb\}\] For some integer $q$. $X$ is nonempty as $a - qb \geq 0$ by choice of $q$. If $a > 0$, we pick $q = 0$. If $a \leq 0$, we pick $q = a$. By WP, $X$ has a minimal element that we will call $r$; $r = a - qb$ for some $q \in \mathbb{Z}$. Since $r \in \mathbb{N} \cup \{0\}$, $r \geq 0$. $r$ also satisfies $r < b$. If $r \geq b$, then $r - b \in X$ as $r - b = (a - qb) - b = a - (q+1)b$. This contradicts minimality of $r$. Rearranging $r = a - qb$ we get $a = qb + r$.

\textit{Uniqueness}: Suppose we have $(q_1, r_1)$ and $(q_2, r_2)$ both satisfying the theorem, WTS $q_1 = q_2$ and $r_1 = r_2$.

We have $a = q_1b + r_1 = q_2b + r_2$ with $r_1, r_2 \in (0, b)$. If we collect the terms with $b$ on one side, we have $(q_1 - q_2)b = r_2 - r_1$. So, $r_2 - r_1$ is a multiple of $b$. Given the constraint $r_1, r_2 \in (0, b)$, we can see that $r_2 - r_1 \in [-(b-1),(b-1)]$. Therefore it is only possible that $r_2 - r_1 = 0$ is a multiple a multiple of $b$. Therefore $r_2 = r_1$ and $(q_1 - q_2)b = 0 \Rightarrow q_1 = q_2$.

\subsection{What if $b < 0$?}

$a = qb + r \Leftrightarrow a = (-q)(-b) + r$. The theorem still works, but $0 \geq r \geq |b|$ needs to be guaranteed.

\section {Divisibility in $\mathbb{Z}$}

\textbf{Definition}: Let $d, a \in \mathbb{Z}$, we say that $d$ divides $a$, written as $d | a$, if $a = qd$ for some $q \in \mathbb{Z}$.

Equivalently: $d$ is a \textit{divisor} of $a$, $a$ is a \textit{multiple} of $d$, or $a$ is \textit{divisible} by $d$.

Some Facts:
\begin{itemize}
    \item $\forall d \in \mathbb{Z}\;\;d | 0$ but $0 \nmid a$ unless $a = 0$.
    \item If $d$ divides $a \neq 0$ then $|d| \leq |a|$. In particular, the set of divisors of a non-zero integer is finite.
    \item $d|a \Leftrightarrow |d| \;|\; |a|$ 
\end{itemize}

\section {GCD in $\mathbb{Z}$}

\textbf{Definition}: Let $a, b \in \mathbb{Z}$, not both 0. The \textit{greatest common divisor} of $a$ and $b$, $\gcd(a, b)$ is the greatest $d \in \mathbb{Z}$ such that $d | a$ and $d | b$.

\textbf{Lemmas}:
\begin{itemize}
    \item $( d | a \land d | b )\rightarrow d|(a-b)$
    \item $( d | (a-b) \land d | b )\rightarrow d|a$    
\end{itemize}

Note that these lemmas mean that if $d$ is a common divisor of $(a, b)$ then it is equivalent to $d$ is a common divisor of $(b, a-b)$; $\gcd(a, b) = gcd(b, a- b)$.

\section{Bezout's Identity in $\mathbb{Z}$}

\textbf{Theorem}:

Let $g = \gcd(a, b)$. Then $g = ax + by$ for some $x, y \in \mathbb{Z}$.

\textbf{Proof}: Suppose we have two sets:
\[\begin{aligned}
D &= \{ d \in \mathbb{Z} \;|\; d|a \land d|b \} \\
I &= \{ ax + by \;|\; x,y \in \mathbb{Z}\}
\end{aligned}\]

$D$ is the set of all common divisors between $a, b$ and $I$ is the set of all integer combinations of $a, b$.

From this we make claim (1): If $d \in D$ and $n \in I$, then $d | n$. In particular, if $n \neq 0$, $|d| \leq |n|$. 

Since $d \in D$, we have $a = q_1d$ and $b = q_2d$ for some $q_1, q_2 \in \mathbb{Z}$. Similarly, since $n \in I$, we have $n = ax + by$ for some $x, y \in \mathbb{Z}$. We can see that $n = ax + by = q_1dx + q_2dy = d(q_1x + q_2y) \Rightarrow d | n$.

Suppose now we look at $I \cap \mathbb{N}$, the integer multiples of $a,b$ that are natural numbers, we let $n^* = \min(I \cap \mathbb{N}) = ax^*+by^*$.

We proceed to make claim (2) that $n^* | a$ and $n^* | a$ (i.e. $n^* \in D$).

Suppose $n^* \nmid a$, we divide $a$ by $n^*$ to get $a = qn^* + r$, $r \in (0, n^*)$. By definition of $n^*$, we see that \[\begin{aligned}
    r &= a - qn^* \\
    &= a - q(ax^* + by^*)\\
    &=a - qax^* + qby^* \\
    &=a(1 - qx^*) + b(qy^*) \\
\end{aligned}\]

This means that $n^* \in I$ and that contradicts the minimality of $n^*$ as $r \in (0, n^*)$.

Finally, we make our last claim (3): $n^* = \max(D) = \gcd(a, b)$. By claim (2), $n^*$ is a common divisor of $a, b$. If $d \in D$ is any other common divisor, then $d \leq n^*$ by claim (1). We can see that $d \leq |d| \leq |n^*| = n^*$.

Therefore, we have two interpretations of $\gcd(a, b)$:
\begin{itemize}
    \item $\gcd(a, b) = \max(D)$
    
    maximal element in set of common divisors
    \item $\gcd(a, b) = \min(I \cap \mathbb{N})$
    
    smallest positive integer combination of $a, b$.
    
\end{itemize}

\section{Euclidean Algorithm}

\textbf{Theorem}:

If $a = qb + r$, then $\gcd(a, b) = \gcd(b, r)$.

\textbf{Proof}:

It is given that $\gcd(a, b) = \gcd(a - b, b)$. As $r = a - qb$, we can consider applying the $a - b$ operation $q$ times: $\gcd(a, b) = \gcd(b, a - qb) = \gcd(b, r)$.

\textbf{Algorithm}

Given: $(a, b)$ with $a > b > 0$ and repeatedly apply division theorem on $(a, b)$. After each division, we replace $a$ with $b$ and $b$ with the remainder of the division:\[\begin{aligned}
    (a, b) \;\;\;& a  = q_1b + r_1 \\
    (b, r_1) \;\;\;& b  = q_2r_1 + r_2 \\
    (r_1, r_2) \;\;\;& r_1  = q_3r_2 + r_3 \\
    \vdots \;\;\;& \vdots \\
    (r_{k-2},r_{k-1}) \;\;\;& r_{k-2} = q_kr_{k-1}+ r_k \\
    (r_{k-1}, r_k) \;\;\;& r_{k-1} = q_{k+1}r_{k}+ 0 \\
    (r_k, 0) \;\;\;& 
\end{aligned}\]

The algorithm stops when we reach a point where the second value in the tuple is $0$, in which case $\gcd(a, b) = r_k$.

This algorithm is guaranteed to terminate as each of the $r_i$ up to terminating $r_k$ are strictly decreasing natural numbers. By WP there is a minimal element to which this procedure will terminate on.

% TODO: talk about how to recover coefficients x, y in ax + by

% CONTINUE: Sept 18_1

\end{multicols*}

\end{document}